\documentclass{article}
\usepackage[utf8]{inputenc}
\usepackage{graphicx}
\usepackage[usestackEOL]{stackengine}
\usepackage{indentfirst}
\usepackage[serbian]{babel}

\date{}
\title{}
\begin{document}
\clearpage\maketitle
\thispagestyle{empty}
\providecommand{\inlinecode}[1]{\texttt{#1}}
\begin{center}
    \huge\textbf{BrainPuzzles}

    \hspace*{1cm}\includegraphics[width=10cm]{Resources/logo.png}\\
    \footnotesize Verzija 1.0

\end{center}
{\raggedleft\vfill\Longstack[l]{%
Veljko Selaković 2019/0331\\
Teodor Delibašić 2019/0316\\
Iva Rakić 2019/0152\\
Lana Jevremović 2019/0345\\
}\par
}
\newpage
\section*{Istorija izmena}
\vspace{1cm}
\begin{center}
\begin{tabular}{||c |c |c |c||} 
 \hline
 Datum & Verzija & Opis & Autor \\ [0.5ex] 
 \hline\hline
 15.3.2022. & 1.0 & Inicijalna verzija & Veljko Selaković \\ 
 \hline
 & & &\\
 \hline
\end{tabular}
\end{center}
\newpage
\tableofcontents

\newpage
\section{Uvod}
Ovaj dokument predstavlja opis projektnog zadatka “BrainPuzzles”. Cilj projekta je raspodela 
aktivnosti slična onoj koja se koristi u realnim softverskim projektima. Osim kratkog opisa problema, biće 
definisane i tehnologije sa kojima se radi, kao i način na koji će biti korišćene.
\subsection{Opis projekta}
\par \textit{BrainPuzzles} je internet aplikacija koja omogućava korisniku da koristi svoj mozak. 
Korisnik pokušava da ostvari što više poena u partiji. Partija se sastoji od 3 igre koje se 
postepeno otključavaju. Tip korisnika određuje koje igre su mu otključane. 
\par Osim regularnih korisnika postoje i administratori koji održavaju sistem i mogu da 
dodaju nova pitanja i teme. U budućnosti bi administratori imali veći nivo kontrole nad 
sistemom. 
\par Postoji globalna rang lista na kojoj se nalaze svi igrači, a prvih deset su javno prikazani na 
odvojenoj stranici.
\newpage
\section{Tehnologije}
\begin{itemize}
    \item Backend
    \begin{itemize}
        \item \inlinecode{PHP} za server-side programiranje
        \item Komunikacija server-client se ostvaruje \inlinecode{AJAX} konceptom
    \end{itemize}
    \item Frontend
    \begin{itemize}
        \item \inlinecode{HTML5, CSS3}
        \item \inlinecode{JavaScript}
        \item \inlinecode{Vue.js} framework \inlinecode{JavaScript}-a za bolji rad na frontend delu
    \end{itemize}
    \item Baza podataka
    \begin{itemize}
        \item \inlinecode{MySQL} kao menadžer baze podataka
    \end{itemize}
\end{itemize}
\newpage
\section{Opis igara}
Trenutni plan je da postoje sledeće tri igre:
\begin{itemize}
    \item Fight list
    \item Mozgić
    \item Ko zna zna
\end{itemize}
\subsection{Fight list}
Cilj igre je nabrojati što više pojmova na zadatu temu. Teme mogu biti raznovrsne, na 
primer: voće, ubijeni američki predsednici, marke šminke... Tema se nasumično bira iz baze na 
početku igre i igrač, odnosno korisnik, ima jedan minut da nabroji što više stvari. Nabrojani 
pojmovi nose različit broj poena u zavisnosti od težine, to jest retkosti odgovora. Najlakši 
odgovor donosi dva poena, srednji po težini četiri, dok najteži vredi šest poena.
\par Pojmovi su u bazi podataka predstavljeni tabelom koja sadrži \inlinecode{
IdTeme i Naziv}, dok se 
pojmovi (odgovori) čuvaju u odvojenoj tabeli koja \inlinecode{IdPojma}, \inlinecode{Naziv} (sam tekst odgovora), 
\inlinecode{Poeni}(broj poena koje taj odgovor donosi korisniku) i \inlinecode{IdTeme} (strani ključ iz tabele Tema; tema 
na koju se odgovor odnosi).
\subsection{Mozgić}
Ova igra je poprilično jednostavna i analogna je igri Skočko iz TV Slagalice. Postoje četiri 
simbola i igrač treba da pronađe tačnu kombinaciju ovih simbola u šest ili manje koraka. U 
zavinosti od broja koraka u kom je korisnik pogodio kombinaciju, dobiće odgovarajući broj 
poena. Ukoliko je pronađena tačna kombinacija u prva četiri koraka igra nosi dvadeset poena, u 
petom nosi petnaest poena, dok se za šesti korak dobija deset poena. Nakon svakog pokušaja, 
korisniku se ispisuje broj simbola koje je stavio na dobru poziciju i broj simbola koji se nalaze u 
tačnoj kombinaciji, ali nisu stavljeni na odgovarajuću poziciju. Kombinaciju treba pogoditi za 
najviše devedeset sekundi od početka igre.
\par Igra ne zahteva čuvanje u bazi podataka jer se kombinacija na početku igre generiše 
nasumično.
\newpage
\subsection{Ko zna zna}
Ovo je igra znanja. Igrač treba da odgovori na deset pitanja koja izlaze jedno po jedno i 
za svako pitanje ima deset sekundi. Korisnik treba sam da ukuca odgovor u predviđeno polje, a 
zatim će se taj odgovor uporediti sa tačnim odgovorom na pitanje. Za tačan odgovor se dobija 
pet poena, dok netačan odgovor donosi dva negativna poena.
\par Pitanje se čuva u tabeli u bazi koja sadrži \inlinecode{IdPitanja, Tekst i Username} (korisnik koji je 
dodao ovo pitanje). Odgovor se takođe čuva u bazi i u toj tabeli nalaze se \inlinecode{IdOdgovora, Tekst i 
IdPitanja}. Iz baze se pitanja izvlače nasumično, a prilikom korisnikovog odgovora se ove dve 
tabele spajaju po atributu \inlinecode{IdPitanja} i porede se odgovori igrača i onaj u tabeli
\newpage
\section{Korisnici}
\subsection{Gost}
Gost je korisnik koji se nije registrovao i može da igra samo prvu igru, odnosno \textit{Fight List}. 
Nema svoj profil, ne pamti mu se high score i ne može se naći na rang listi.
\subsection{Bronzani korisnik}
Gost koji se prvi put registruje automatski postaje Bronzani korisnik koji dalje može da 
napreduje ka Srebrnom i Zlatnom. Ima svoj profil i pristup samo \textit{Fight List} igri. Prilikom svakog 
ulaska na aplikaciju se može ulogovati kako bi mu se pamtio High score, Total score, bio na rang 
listi i kasnije otključavao i ostale igrice. 
\subsection{Srebrni korisnik}
Bronzani korisnik igranjem prve igre skuplja poene, na kraju svake igre poeni se beleže u 
njegov Total score i kad dostigne određeni broj poena postaje Srebrni. Njemu se otključava i 
druga igra, \textit{Mozgić}. 
\subsection{Zlatni korisnik}
Prilikom ostvarivanja određenog broja poena, Srebrni Korisnik se unapređuje u Zlatnog i 
time mu se omogućava i pristup igri \textit{Ko zna zna}. 
\subsection{Administrator}
Registrovani korisnik koji osim prava da igra igre, može da dodaje nova pitanja u bazu za 
prvu i treću igru i dodeli ulogu admina drugom registrovanom korisniku
\newpage
\section{Rang lista}
Samo registrovani korisnici (\textit{Bronzani, Srebrni i Zlatni}) se mogu naći na rang listi. Rangiraju se na osnovu 
high score-a. Korisnicima se prikazuje prvih 10 mesta rang liste.
\section{Funkcionalnosti}
\begin{itemize}
    \item \textbf{Registracija} \\
    Kada gost želi da se registruje, potrebno je da ostavi svoje ime, email, i smisli svoje 
korisničko ime i lozinku. Od tog trenutka on ima svoj profil na koji se može logovati.
    \item \textbf{Logovanje} \\
    Nakon što se korisnik prvi put registrovao prilikom svakog ulaska u aplikaciju unošenjem 
svog korisničkog imena i lozinke se loguje
    \item \textbf{Pokretanje partije} \\
    Korisnik pokreće partiju koja se sastoji od tri gorepomenute igre

    \item \textbf{Pokretanje igre} \\
    Korisnik pokreće jednu od specifičnih igara koje čine jednu partiju

    \item \textbf{Submitovanje rezultata} \\
    Akcija koju korisnik preduzima na kraju igre
    \item \textbf{Pregledanje prvih 10 korisnika na rang listi} \\
    Korisnik može da pogleda ko su trenutno najbolji takmičari i da se poredi sa njima
    \item \textbf{Pregledanje ličnog profila} \\
    Dostupna je statistika profila, kao i dodatne funkcionalnosti koje čine profil jedinstvenim 
poput \textit{About me} sekcije
\end{itemize}


\end{document}
